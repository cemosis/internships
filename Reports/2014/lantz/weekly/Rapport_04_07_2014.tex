\documentclass[12pt]{article}
 
\usepackage[utf8]{inputenc}
\usepackage[T1]{fontenc}
\usepackage[french]{babel}
\usepackage{enumerate}
\usepackage{listingsutf8}
\usepackage{xcolor}
\usepackage{graphicx}
\usepackage{amssymb}
\usepackage{amsmath}
\usepackage{geometry}
\usepackage{float}
\usepackage{tikz}
\usepackage{url}



\geometry {
	left = 2.5cm,
	right = 2.5cm,
	bottom = 2.5cm ,
	top = 2.5cm ,
}

\lstdefinestyle{customjava}{
 belowcaptionskip=1\baselineskip,
 breaklines=true,
 frame=single,
 %linewidth=7.5cm,
 framexleftmargin=5mm,
 %frameround=tttt,
 %framexrightmargin=5mm,
 xleftmargin=\parindent,
 language=c++,
 showstringspaces=false,
 basicstyle=\footnotesize\ttfamily,
 keywordstyle=\color{green!40!black},
 ndkeywordstyle=\color{orange},
 commentstyle=\color{purple!40!black},
 identifierstyle=\color{blue},
 stringstyle=\color{red},
 numbers=left,
 numbersep=7pt,
 inputencoding=utf8
}

\frenchbsetup{StandardLists=true}
\title {Rapport : Wrapping Python}
\author {Thomas \emph{LANTZ}}

\lstset{style=customjava, emph={int,double,void, Double}, emphstyle=\color{red}, emph={[2]wavJava, Spectrum}, emphstyle=[2]\color{orange}}

\begin{document}
\maketitle 

\section{Introduction}

Pendant cette cinquième semaine de stage, je me suis attaqué à la finalisation de wrapping de la classe Environment, afin de pouvoir enfin passer sur les classes et méthodes suivantes que sont Simplex/Hypercube, Mesh, loadMesh.

Notre but étant à l'heure actuelle de pouvoir réécrire le programme contenu dans mymesh.cpp
(dans ~/feelpp/doc/manual/tutorial) dans un script Python.

\begin{lstlisting}
int main( int argc, char** argv )
{
    // initialize Feel++ Environment
    Environment env( _argc=argc, _argv=argv,
                     _about=about( _name="mymesh" ,
                                   _author="Feel++ Consortium",
                                   _email="feelpp-devel@feelpp.org" ) );
    //! [load]
    // create a mesh with GMSH using Feel++ geometry tool
    auto mesh = loadMesh(_mesh=new  Mesh<Simplex<2>>);
    //! [load]

#if 1
    //! [export]
    // export results for post processing
    auto e = exporter( _mesh=mesh );
    e->addRegions();
    e->save();
    //! [export]
#endif
}   // main
//! [all]
\end{lstlisting}

\section{Wrapping de la classe Environment}

Après avoir enfin réussi à corriger une erreur commise de mon propre chef (Copie des fichiers environment pour travailler dessus, ce qui provoquait des soucis à l'intégration dans la bibliothèque Feel), j'ai pu enfin terminer de wrapper la classe Environment et l'utiliser dans un script Python.

Pour se faire, je suis partit du constructeur, prenant un ArgumentPack en paramètre et qui est appelé dans la BOOST\_PARAMETER\_FUNCTION de Environment, présent ci-dessous :
\newline
\begin{lstlisting}
template <class ArgumentPack>
    Environment(ArgumentPack const& args)
        {
            char** argv = args[_argv];
            int argc = args[_argc];
            S_desc_app = boost::shared_ptr<po::options_description>( new po::options_description( args[_desc|Feel::feel_nooptions()] ) );
            S_desc_lib = boost::shared_ptr<po::options_description>( new po::options_description( args[_desc_lib | Feel::feel_options()] ) );
            AboutData about = args[_about| makeAbout(argv[0])];
            S_desc = boost::shared_ptr<po::options_description>( new po::options_description( ) );
            S_desc->add( *S_desc_app );

            // try to see if the feel++ lib options are already in S_desc_app, if yes then we do not add S_desc_lib
            // otherwise we will have duplicated options
            std::vector<boost::shared_ptr<po::option_description>> opts = Environment::optionsDescriptionApplication().options();
            auto it = std::find_if( opts.begin(), opts.end(),
                                    []( boost::shared_ptr<po::option_description> const&o )
                                    {
                                        return o->format_name().erase(0,2) == "backend";
                                    });

            if   ( it == opts.end() )
                S_desc->add( *S_desc_lib );
            S_desc->add( file_options( about.appName() ) );
            S_desc->add( generic_options() );


            init( argc, argv, *S_desc, *S_desc_lib, about );
            if ( S_vm.count("nochdir") == 0 )
            {
                std::string defaultdir = about.appName();
                if ( S_vm.count("directory") )
                    defaultdir = S_vm["directory"].as<std::string>();
                std::string d = args[_directory|defaultdir];
                LOG(INFO) << "change directory to " << d << "\n";
                boost::format f( d );
                changeRepository( _directory=f );
            }
        }

\end{lstlisting}

Je l'ai alors modifié afin de recréer à coté un constructeur similaire mais avec une Boost$::$python$::$list en paramètre. Ce qui nous donne :

\begin{lstlisting}
Environment::Environment(boost::python::list arg)
{
    
    int argc = boost::python::len(arg);
           std::cout << argc << std::endl ;

           char** argv =new char* [argc+1];
           boost::python::stl_input_iterator<std::string> begin(arg), end;
           int i=0;
           while (begin != end)
           {
           std::cout << *begin << std::endl ;
           argv[i] =strdup((*begin).c_str());
           begin++;
           i++;
           }
           argv[argc]=NULL;

    S_desc_app = boost::shared_ptr<po::options_description>( new po::options_description(Feel::feel_nooptions() ) );
    S_desc_lib = boost::shared_ptr<po::options_description>( new po::options_description( Feel::feel_options() ) );
    AboutData about =makeAbout(argv[0]);
    S_desc = boost::shared_ptr<po::options_description>( new po::options_description( ) );
    S_desc->add( *S_desc_app );

    // try to see if the feel++ lib options are already in S_desc_app, if yes then we do not add S_desc_lib
    // otherwise we will have duplicated options
    std::vector<boost::shared_ptr<po::option_description>> opts = Environment::optionsDescriptionApplication().options();
    auto it = std::find_if( opts.begin(), opts.end(),
            []( boost::shared_ptr<po::option_description> const&o )
            {
            return o->format_name().erase(0,2) == "backend";
            });

    if   ( it == opts.end() )
        S_desc->add( *S_desc_lib );
    S_desc->add( file_options( about.appName() ) );
    S_desc->add( generic_options() );


    init( argc, argv, *S_desc, *S_desc_lib, about );
    if ( S_vm.count("nochdir") == 0 )
    {
        std::string defaultdir = about.appName();
        if ( S_vm.count("directory") )
            defaultdir = S_vm["directory"].as<std::string>();
        std::string d =defaultdir;
        LOG(INFO) << "change directory to " << d << "\n";
        boost::format f( d );
        changeRepository( _directory=f );
    }
}
\end{lstlisting}

On réutilise ici un morceau de la méthode wrap qui permettait à partir d'une liste Python de récupérer les arguments argc et argv. Pour les autres champs à remplir, on y place simplement les valeurs par défauts.

Le BOOST\_PYTHON\_MODULE donne alors :

\begin{lstlisting}
BOOST_PYTHON_MODULE(libEnvironment)
{
   if (import_mpi4py()<0) return ;
  
    class_<Argv>("Argv",init<boost::python::list>())
       .def("create",&Argv::create,return_value_policy<manage_new_object>())
       .def("test",&Argv::test);

    class_<Feel::detail::Environment,boost::noncopyable>("Environment", init<boost::python::list>());
}
\end{lstlisting}

On peut alors l'appeler dans un script Python de la manière suivante :
\begin{lstlisting}
#!/usr/bin/python

from mpi4py import MPI
import libEnvironment
import sys

x=libEnvironment.Environment(sys.argv)
\end{lstlisting}

\section{Wrapping de la classe Simplex et de Mesh}

Afin de pouvoir continuer sur notre programme, nous avons besoin de pouvoir récupérer des objets de type Mesh et donc des objets de type Simplex/Hypercube nécessaire à sa construction.

On peut alors écrire un BOOST\_PYTHON\_MODULE de la sorte :
\begin{lstlisting}
#include <feel/feel.hpp>
#include <boost/python.hpp>
#include <boost/python/stl_iterator.hpp>
#include <mpi4py/mpi4py.h>

#include <boost/parameter/keyword.hpp>
#include <boost/parameter/preprocessor.hpp>
#include <boost/parameter/python.hpp>
#include <boost/python.hpp>
#include <boost/mpl/vector.hpp>

using namespace boost::python;
using namespace Feel;

namespace py = boost::parameter::python;

BOOST_PYTHON_MODULE(libMesh)
{
   
   if (import_mpi4py()<0) return ;
       
   class_<Feel::Simplex<2>>("Simplex",init<>())
    .def("dim",&Feel::Simplex<2>::dimension);
   
   
   class_<Feel::Mesh<Feel::Simplex<2>>,boost::noncopyable>("Mesh",init<>());

    
}
\end{lstlisting}

On pourra ici remarquer que afin de wrapper la classe Simplex ou Mesh, on doit fournir des valeurs liées au template de la classe. On ne peut pas faire un wrapp générique de la classe, ce qui peut poser problème pour de futurs applications.

Pour la classe Hypercube,il suffit de changer la partie Simplex en Hypercube.

On obtient alors un script Python de la forme :

\begin{lstlisting}

#!/usr/bin/python

from mpi4py import MPI
import libEnvironment
import libMesh
import sys

x=libEnvironment.Environment(sys.argv)

k=libMesh.Simplex()
print k.dim()

m=libMesh.Mesh();
\end{lstlisting}


\section{Wrapping de la méthode loadMesh}

Maintenant que Mesh et Simplex ont été "wrappés", on veut s'intéresser à la méthode loadMesh qui permet de charger le maillage donné en argument. Pour se faire,on doit wrapper la BOOST\_PARAMETER\_FUNCTION suivante :


\begin{lstlisting}
BOOST_PARAMETER_FUNCTION(
    ( typename Feel::detail::mesh<Args>::ptrtype ), // return type
    loadMesh,    // 2. function name

    tag,           // 3. namespace of tag types

    ( required
      ( mesh, *)

        ) // 4. one required parameter, and

    ( optional
      ( filename, *( boost::is_convertible<mpl::_,std::string> ), option(_name="gmsh.filename").template as<std::string>() )
      ( desc, *,boost::shared_ptr<gmsh_type>() )  // geo() can't be used here as default !!

      ( h,              *( boost::is_arithmetic<mpl::_> ), option(_name="gmsh.hsize").template as<double>() )
      ( straighten,          (bool), option(_name="gmsh.straighten").template as<bool>() )
      ( refine,          *( boost::is_integral<mpl::_> ), option(_name="gmsh.refine").template as<int>() )
      ( update,          *( boost::is_integral<mpl::_> ), MESH_CHECK|MESH_UPDATE_FACES|MESH_UPDATE_EDGES )
      ( physical_are_elementary_regions,		   (bool), option(_name="gmsh.physical_are_elementary_regions").template as<bool>() )
      ( worldcomm,       (WorldComm), Environment::worldComm() )
      ( force_rebuild,   *( boost::is_integral<mpl::_> ), option(_name="gmsh.rebuild").template as<bool>() )
      ( respect_partition,	(bool), option(_name="gmsh.respect_partition").template as<bool>() )
      ( rebuild_partitions,	(bool), option(_name="gmsh.partition").template as<bool>() )
      ( rebuild_partitions_filename, *( boost::is_convertible<mpl::_,std::string> )	, filename )
      ( partitions,      *( boost::is_integral<mpl::_> ), worldcomm.globalSize() )
      ( partitioner,     *( boost::is_integral<mpl::_> ), option(_name="gmsh.partitioner").template as<int>() )
      ( partition_file,   *( boost::is_integral<mpl::_> ), 0 )
      ( depends, *( boost::is_convertible<mpl::_,std::string> ), option(_name="gmsh.depends").template as<std::string>() )
        )
    )
{
#if defined(__clang__)
#pragma clang diagnostic push
#pragma clang diagnostic ignored "-Wunsequenced"
#endif
    typedef typename Feel::detail::mesh<Args>::type _mesh_type;
    typedef typename Feel::detail::mesh<Args>::ptrtype _mesh_ptrtype;

    // look for mesh_name in various directories (executable directory, current directory. ...)
    // return an empty string if the file is not found

    fs::path mesh_name=fs::path(Environment::findFile(filename));
    LOG_IF( WARNING, mesh_name.extension() != ".geo" && mesh_name.extension() != ".msh" )
        << "Invalid filename " << filename << " it should have either the .geo or .msh extension\n";


    if ( mesh_name.extension() == ".geo" )
    {

        return createGMSHMesh(
            _mesh=mesh,
            _desc= (!desc) ? geo( _filename=mesh_name.string(),
                                  _h=h,
                                  _depends=depends,
                                  _worldcomm=worldcomm  ) : desc ,
            _h=h,
            _straighten=straighten,
            _refine=refine,
            _update=update,
            _physical_are_elementary_regions=physical_are_elementary_regions,
            _force_rebuild=force_rebuild,
            _worldcomm=worldcomm,
            _respect_partition=respect_partition,
            _rebuild_partitions=rebuild_partitions,
            _rebuild_partitions_filename=rebuild_partitions_filename,
            _partitions=partitions,
            _partitioner=partitioner,
            _partition_file=partition_file
            );
    }

    if ( mesh_name.extension() == ".msh"  )
    {
        return loadGMSHMesh( _mesh=mesh,
                             _filename=mesh_name.string(),
                             _straighten=straighten,
                             _refine=refine,
                             _update=update,
                             _physical_are_elementary_regions=physical_are_elementary_regions,
                             _worldcomm=worldcomm,
                             _respect_partition=respect_partition,
                             _rebuild_partitions=rebuild_partitions,
                             _rebuild_partitions_filename=rebuild_partitions_filename,
                             _partitions=partitions,
                             _partitioner=partitioner,
                             _partition_file=partition_file
            );

    }

    LOG(WARNING) << "File " << mesh_name << " not found, generating instead an hypercube in " << _mesh_type::nDim << "D geometry and mesh...";
    return createGMSHMesh(_mesh=mesh,
                          _desc=domain( _name=option(_name="gmsh.domain.shape").template as<std::string>(), _h=h ),
                          _h=h,
                          _refine=refine,
                          _update=update,
                          _physical_are_elementary_regions=physical_are_elementary_regions,
                          _force_rebuild=force_rebuild,
                          _worldcomm=worldcomm,
                          _respect_partition=respect_partition,
                          _rebuild_partitions=rebuild_partitions,
                          _rebuild_partitions_filename=rebuild_partitions_filename,
                          _partitions=partitions,
                          _partitioner=partitioner,
                          _partition_file=partition_file );
#if defined(__clang__)
#pragma clang diagnostic pop
#endif
} // loadMesh
\end{lstlisting}

On va alors utiliser la méthode présente dans le lien suivant afin de wrapper cette méthode:
\url{http://www.boost.org/doc/libs/1_55_0/libs/parameter/doc/html/python.html#class-template-init}

Ce qui nous donne :
\begin{lstlisting}
struct loadMesh_fwd
{
    template<class A0,class A1,class A2,class A3,class A4,class A5,class A6,class A7,class A8,class A9,class A10,class A11,class A12,class A13,class A14,class A15>
void operator() ( 
        boost::type<Feel::detail::mesh<Args>::ptrtype>, &self, A0 const& a0, A1 const& a1,A2 const& a2,A3 const& a3,A4 const& a4,A5 const& a5,A6 const& a6,A7 const& a7,A8 const& a8,A9 const& a9,A10 const& a10,A11 const& a11,A12 const& a12,A13 const& a13,A14 const& a14,A15 const& a15
        )
    {
        self.loadMesh(a0,a1,a2,a3,a4,a5,a6,a7,a8,a9,a10,a11,a12,a13,a14,a15);
    }
};
\end{lstlisting}

\newpage

\begin{lstlisting}
BOOST_PYTHON_MODULE(libMesh)
{
   
   if (import_mpi4py()<0) return ;
       
   class_<Feel::Simplex<2>>("Simplex",init<>())
    .def("dim",&Feel::Simplex<2>::dimension);
   
   
   class_<Feel::Mesh<Feel::Simplex<2>>,boost::noncopyable>("Mesh",init<>());

    
    def(
        "loadMesh",py::function<
                loadMesh_fwd
                ,mpl::vector<
                    Feel::detail::mesh<Args>::ptrtype
                    , tag::mesh(*)
                    , tag::filename*(*(boost::is_convertible<mpl::_,std::string>))
                    , tag::desc*(*)
                    , tag::h*(*(boost::is_arithmetic<mpl::_>))
                    , tag::straighten*(bool)
                    , tag::refine*(*(boost::is_integral<mpl::_>))
                    , tag::update*(*(boost::is_integral<mpl::_>))
                    , tag::physical_are_elementary_regions*(bool)
                    , tag::worldcomm*(WorldComm)
                    , tag::force_rebuild*(*(boost::is_integral<mpl::_>))
                    , tag::respect_partition*(bool)
                    , tag::rebuild_partitions*(bool)
                    , tag::partitions*(*(boost::is_integral<mpl::_>))
                    , tag::partitioner*(*(boost::is_integral<mpl::_>))
                    , tag::partition_file*(*(boost::is_integral<mpl::_>))
                    , tag::depends*(*(boost::is_convertible<mpl::_,std::string>))
                    >
           >()
     );
}
\end{lstlisting}

Malheuresement ce n'est pas encore au point.
\section{Travaux en cours}
- Suite du travail sur la méthode loadMesh

- Recherche et documentation sur la suite des classes et méthodes ( exporter,....)

\section{Problèmes rencontrés en cours de semaine}
- Erreur suite à la duplication de la classe Environment qui causait des soucis lors de la création du module associé.

- Impossibilité de faire un wrapping général des classes Simplex, Hypercube et Mesh.

- Tentatives de wrapping de la méthode loadMesh non concluantes
\end{document}
	