\documentclass[12pt]{article}
 
\usepackage[utf8]{inputenc}
\usepackage[T1]{fontenc}
\usepackage[french]{babel}
\usepackage{enumerate}
\usepackage{listingsutf8}
\usepackage{xcolor}
\usepackage{graphicx}
\usepackage{amssymb}
\usepackage{amsmath}
\usepackage{geometry}
\usepackage{float}
\usepackage{tikz}
\usepackage{url}



\geometry {
	left = 2.5cm,
	right = 2.5cm,
	bottom = 2.5cm ,
	top = 2.5cm ,
}

\lstdefinestyle{customjava}{
 belowcaptionskip=1\baselineskip,
 breaklines=true,
 frame=single,
 %linewidth=7.5cm,
 framexleftmargin=5mm,
 %frameround=tttt,
 %framexrightmargin=5mm,
 xleftmargin=\parindent,
 language=c++,
 showstringspaces=false,
 basicstyle=\footnotesize\ttfamily,
 keywordstyle=\color{green!40!black},
 ndkeywordstyle=\color{orange},
 commentstyle=\color{purple!40!black},
 identifierstyle=\color{blue},
 stringstyle=\color{red},
 numbers=left,
 numbersep=7pt,
 inputencoding=utf8
}

\frenchbsetup{StandardLists=true}
\title {Rapport : Wrapping Python}
\author {Thomas \emph{LANTZ}}

\lstset{style=customjava, emph={int,double,void, Double}, emphstyle=\color{red}, emph={[2]wavJava, Spectrum}, emphstyle=[2]\color{orange}}

\begin{document}
\maketitle 

\section{Introduction}

Durant cette sixième semaine en stage,j'ai concentré mes efforts sur les 2 méthodes principales restantes à savoir loadMesh et exporter afin de pouvoir finaliser la construction du script reprenant mymesh.cpp.


\section{Wrapping de la méthode loadMesh}

Retour sur la méthode loadMesh qui permet de charger le maillage donné en argument. Pour se faire,on voudrait wrapper la BOOST\_PARAMETER\_FUNCTION suivante :


\begin{lstlisting}
BOOST_PARAMETER_FUNCTION(
    ( typename Feel::detail::mesh<Args>::ptrtype ), // return type
    loadMesh,    // 2. function name

    tag,           // 3. namespace of tag types

    ( required
      ( mesh, *)

        ) // 4. one required parameter, and

    ( optional
      ( filename, *( boost::is_convertible<mpl::_,std::string> ), option(_name="gmsh.filename").template as<std::string>() )
      ( desc, *,boost::shared_ptr<gmsh_type>() )  // geo() can't be used here as default !!

      ( h,              *( boost::is_arithmetic<mpl::_> ), option(_name="gmsh.hsize").template as<double>() )
      ( straighten,          (bool), option(_name="gmsh.straighten").template as<bool>() )
      ( refine,          *( boost::is_integral<mpl::_> ), option(_name="gmsh.refine").template as<int>() )
      ( update,          *( boost::is_integral<mpl::_> ), MESH_CHECK|MESH_UPDATE_FACES|MESH_UPDATE_EDGES )
      ( physical_are_elementary_regions,		   (bool), option(_name="gmsh.physical_are_elementary_regions").template as<bool>() )
      ( worldcomm,       (WorldComm), Environment::worldComm() )
      ( force_rebuild,   *( boost::is_integral<mpl::_> ), option(_name="gmsh.rebuild").template as<bool>() )
      ( respect_partition,	(bool), option(_name="gmsh.respect_partition").template as<bool>() )
      ( rebuild_partitions,	(bool), option(_name="gmsh.partition").template as<bool>() )
      ( rebuild_partitions_filename, *( boost::is_convertible<mpl::_,std::string> )	, filename )
      ( partitions,      *( boost::is_integral<mpl::_> ), worldcomm.globalSize() )
      ( partitioner,     *( boost::is_integral<mpl::_> ), option(_name="gmsh.partitioner").template as<int>() )
      ( partition_file,   *( boost::is_integral<mpl::_> ), 0 )
      ( depends, *( boost::is_convertible<mpl::_,std::string> ), option(_name="gmsh.depends").template as<std::string>() )
        )
    )
{
#if defined(__clang__)
#pragma clang diagnostic push
#pragma clang diagnostic ignored "-Wunsequenced"
#endif
    typedef typename Feel::detail::mesh<Args>::type _mesh_type;
    typedef typename Feel::detail::mesh<Args>::ptrtype _mesh_ptrtype;

    // look for mesh_name in various directories (executable directory, current directory. ...)
    // return an empty string if the file is not found

    fs::path mesh_name=fs::path(Environment::findFile(filename));
    LOG_IF( WARNING, mesh_name.extension() != ".geo" && mesh_name.extension() != ".msh" )
        << "Invalid filename " << filename << " it should have either the .geo or .msh extension\n";


    if ( mesh_name.extension() == ".geo" )
    {

        return createGMSHMesh(
            _mesh=mesh,
            _desc= (!desc) ? geo( _filename=mesh_name.string(),
                                  _h=h,
                                  _depends=depends,
                                  _worldcomm=worldcomm  ) : desc ,
            _h=h,
            _straighten=straighten,
            _refine=refine,
            _update=update,
            _physical_are_elementary_regions=physical_are_elementary_regions,
            _force_rebuild=force_rebuild,
            _worldcomm=worldcomm,
            _respect_partition=respect_partition,
            _rebuild_partitions=rebuild_partitions,
            _rebuild_partitions_filename=rebuild_partitions_filename,
            _partitions=partitions,
            _partitioner=partitioner,
            _partition_file=partition_file
            );
    }

    if ( mesh_name.extension() == ".msh"  )
    {
        return loadGMSHMesh( _mesh=mesh,
                             _filename=mesh_name.string(),
                             _straighten=straighten,
                             _refine=refine,
                             _update=update,
                             _physical_are_elementary_regions=physical_are_elementary_regions,
                             _worldcomm=worldcomm,
                             _respect_partition=respect_partition,
                             _rebuild_partitions=rebuild_partitions,
                             _rebuild_partitions_filename=rebuild_partitions_filename,
                             _partitions=partitions,
                             _partitioner=partitioner,
                             _partition_file=partition_file
            );

    }

    LOG(WARNING) << "File " << mesh_name << " not found, generating instead an hypercube in " << _mesh_type::nDim << "D geometry and mesh...";
    return createGMSHMesh(_mesh=mesh,
                          _desc=domain( _name=option(_name="gmsh.domain.shape").template as<std::string>(), _h=h ),
                          _h=h,
                          _refine=refine,
                          _update=update,
                          _physical_are_elementary_regions=physical_are_elementary_regions,
                          _force_rebuild=force_rebuild,
                          _worldcomm=worldcomm,
                          _respect_partition=respect_partition,
                          _rebuild_partitions=rebuild_partitions,
                          _rebuild_partitions_filename=rebuild_partitions_filename,
                          _partitions=partitions,
                          _partitioner=partitioner,
                          _partition_file=partition_file );
#if defined(__clang__)
#pragma clang diagnostic pop
#endif
} // loadMesh
\end{lstlisting}

On va alors utiliser la méthode présente dans le lien suivant afin de wrapper cette méthode:
\url{http://www.boost.org/doc/libs/1_55_0/libs/parameter/doc/html/python.html#class-template-init}

Ce qui nous donne :
\begin{lstlisting}
struct loadMesh_fwd
{
    template<class A0,class A1,class A2,class A3,class A4,class A5,class A6,class A7,class A8,class A9,class A10,class A11,class A12,class A13,class A14,class A15>
void operator() ( 
        boost::type<Feel::detail::mesh<Args>::ptrtype>, &self, A0 const& a0, A1 const& a1,A2 const& a2,A3 const& a3,A4 const& a4,A5 const& a5,A6 const& a6,A7 const& a7,A8 const& a8,A9 const& a9,A10 const& a10,A11 const& a11,A12 const& a12,A13 const& a13,A14 const& a14,A15 const& a15
        )
    {
        self.loadMesh(a0,a1,a2,a3,a4,a5,a6,a7,a8,a9,a10,a11,a12,a13,a14,a15);
    }
};
\end{lstlisting}



A partir d'ici, j'ai fait plusieurs tentatives de wrapping qui se sont soldés par des échecs

\subsection{Premier essai}


\begin{lstlisting} 

struct exporter_fwd
   {
   template <class A0,class A1,class A2,class A3,class A4,class A5,typename Args>
   typename Feel::detail::compute_exporter_return<Args>::ptrtype
   operator() 
   (boost::type<typename Feel::detail::compute_exporter_return<Args>::ptrtype>
   , A0 const& a0, A1 const& a1 , A2 const& a2 , A3 const& a3 , A4 const& a4, A5 const& a5)
   {
   exporter(a0,a1,a2,a3,a4,a5);
   }
   };

BOOST_PYTHON_MODULE(libMesh)
{
   
   if (import_mpi4py()<0) return ;
       
   class_<Feel::Simplex<2>>("Simplex",init<>())
    .def("dim",&Feel::Simplex<2>::dimension);
   
   
   class_<Feel::Mesh<Feel::Simplex<2>>,boost::noncopyable>("Mesh",init<>());

    
    def(
        "loadMesh",py::function<
                loadMesh_fwd
                ,mpl::vector<
                    Feel::detail::mesh<Args>::ptrtype
                    , tag::mesh(*)
                    , tag::filename*(*(boost::is_convertible<mpl::_,std::string>))
                    , tag::desc*(*)
                    , tag::h*(*(boost::is_arithmetic<mpl::_>))
                    , tag::straighten*(bool)
                    , tag::refine*(*(boost::is_integral<mpl::_>))
                    , tag::update*(*(boost::is_integral<mpl::_>))
                    , tag::physical_are_elementary_regions*(bool)
                    , tag::worldcomm*(WorldComm)
                    , tag::force_rebuild*(*(boost::is_integral<mpl::_>))
                    , tag::respect_partition*(bool)
                    , tag::rebuild_partitions*(bool)
                    , tag::partitions*(*(boost::is_integral<mpl::_>))
                    , tag::partitioner*(*(boost::is_integral<mpl::_>))
                    , tag::partition_file*(*(boost::is_integral<mpl::_>))
                    , tag::depends*(*(boost::is_convertible<mpl::_,std::string>))
                    >
           >()
     );
}
\end{lstlisting}

\subsection{Second essai}

Avec un changement de type de retour 
\begin{lstlisting}
       def(
       "loadMesh",
       py::function<
       loadMesh_fwd
       ,mpl::vector<
       boost::shared_ptr<Mesh<Simplex<2>>>
       , tag::mesh(*(Mesh<Simplex<2>>))
       , tag::filename*(*(boost::is_convertible<mpl::_,std::string>))
       , tag::desc*(Feel::po::options_description const&)
       , tag::h*(*(boost::is_arithmetic<mpl::_>))
       , tag::straighten*(bool)
       , tag::refine*(*(boost::is_integral<mpl::_>))
       , tag::update*(*(boost::is_integral<mpl::_>))
       , tag::physical_are_elementary_regions*(bool)
       , tag::worldcomm*(WorldComm)
       , tag::force_rebuild*(*(boost::is_integral<mpl::_>))
       , tag::respect_partition*(bool)
       , tag::rebuild_partitions*(bool)
       , tag::rebuild_partitions_filename*(*(boost::is_convertible<mpl::_,std::string>))
       , tag::partitions*(*(boost::is_integral<mpl::_>))
       , tag::partitioner*(*(boost::is_integral<mpl::_>))
       , tag::partition_file*(*(boost::is_integral<mpl::_>))
       , tag::depends*(*(boost::is_convertible<mpl::_,std::string>))

       >
       >()
       );
\end{lstlisting}
 

\subsection{Troisième essai }
Avec une méthode def présenté dans le document donné précédemment :
\begin{lstlisting}
def<
       loadMesh_fwd
       ,mpl::vector<
    //Feel::detail::mesh<Mesh<Simplex<2>>>::ptrtype
    Mesh<Simplex<2>>
    , tag::mesh(Mesh<Simplex<2>>)
    , tag::filename*(*(boost::is_convertible<mpl::_,std::string>))
    , tag::desc*(Feel::po::options_description const&)
    , tag::h*(*(boost::is_arithmetic<mpl::_>))
    , tag::straighten*(bool)
    , tag::refine*(*(boost::is_integral<mpl::_>))
    , tag::update*(*(boost::is_integral<mpl::_>))
    , tag::physical_are_elementary_regions*(bool)
    , tag::worldcomm*(WorldComm)
    , tag::force_rebuild*(*(boost::is_integral<mpl::_>))
    , tag::respect_partition*(bool)
    , tag::rebuild_partitions*(bool)
    , tag::rebuild_partitions_filename*(*(boost::is_convertible<mpl::_,std::string>))
    , tag::partitions*(*(boost::is_integral<mpl::_>))
    , tag::partitioner*(*(boost::is_integral<mpl::_>))
    , tag::partition_file*(*(boost::is_integral<mpl::_>))
    , tag::depends*(*(boost::is_convertible<mpl::_,std::string>))

    >
    >("loadMesh");
\end{lstlisting}

\newpage
\subsection{Quatrième essai}

Avec d'autres types ...
\begin{lstlisting}
def(
       "loadMesh",
       py::function<
       loadMesh_fwd
       ,mpl::vector<
    // Feel::detail::mesh<Mesh<Simplex<2>>>::ptrtype
    Mesh<Simplex<2>>
    , tag::mesh(Mesh<Simplex<2>>)
    , tag::filename*(std::string)
    , tag::desc*(Feel::po::options_description const&)
    , tag::h*(double)
    , tag::straighten*(bool)
    , tag::refine*(int)
    , tag::update*(int)
    , tag::physical_are_elementary_regions*(bool)
    , tag::worldcomm*(WorldComm)
    , tag::force_rebuild*(bool)
    , tag::respect_partition*(bool)
    , tag::rebuild_partitions*(bool)
    , tag::rebuild_partitions_filename*(bool)
    , tag::partitions*(int)
    , tag::partitioner*(int)
    , tag::partition_file*(int)
    , tag::depends*(std::string)
    >
    >()
    );
\end{lstlisting}

Malheureusement, à la compilation, il ne semble pas reconnaitre la présence d'une méthode pouvant correspondre à cela.

Suite à cela, je me suis lancé dans le wrapping de la méthode exporter en parallèle de loadMesh , afin de voir si j'obtenais la même erreur avec une méthode qui semblait plus simple.

\section{Wrapping de la méthode exporter }

Dans la même idée que loadMesh, on veut wrapper la BOOST\_PARAMETER\_FUNCTION suivante :

\begin{lstlisting}
BOOST_PARAMETER_FUNCTION( ( typename Feel::detail::compute_exporter_return<Args>::ptrtype ),
                          exporter,                                       // 2. name of the function template
                          tag,                                        // 3. namespace of tag types
                          ( required                                  // 4. one required parameter, and
                            ( mesh, * )
                          ) // required
                          ( optional                                  // 4. one required parameter, and
                            ( fileset, *, option(_name="exporter.fileset").template as<bool>() )
                            ( order, *, mpl::int_<1>() )
                            ( name,  *, Environment::about().appName() )
                            ( geo,   *, option(_name="exporter.geometry").template as<std::string>() )
                            ( path, *( boost::is_convertible<mpl::_,std::string> ), std::string(".") )
                          ) )
{
    typedef typename Feel::detail::compute_exporter_return<Args>::type exporter_type;
    auto e =  exporter_type::New( Environment::vm(),name,mesh->worldComm() );
    e->setPrefix( name );
    e->setUseSingleTransientFile( fileset );
    if ( geo == "change_coords_only" )
        e->setMesh( mesh, EXPORTER_GEOMETRY_CHANGE_COORDS_ONLY );
    else if ( geo == "change" )
        e->setMesh( mesh, EXPORTER_GEOMETRY_CHANGE );
    else // default
        e->setMesh( mesh, EXPORTER_GEOMETRY_STATIC );
    e->setPath( path );
    // addRegions not work with transient simulation!
    //e->addRegions();
    return e;
    //return Exporter<Mesh<Simplex<2> >,1>::New();
}
\end{lstlisting}

On  essaie alors de construire une structure comme précédemment :
\begin{lstlisting}
struct exporter_fwd
   {
   template <class A0,class A1,class A2,class A3,class A4,class A5>
//typename Feel::detail::compute_exporter_return<Args>::ptrtype
std::shared_ptr<Exporter<Mesh<Simplex<2> >,1>>
operator() 
(//boost::type<typename Feel::detail::compute_exporter_return<Args>::ptrtype>
boost::type<std::shared_ptr<Exporter<Mesh<Simplex<2>>,1>>>
, A0 const& a0, A1 const& a1 , A2 const& a2 , A3 const& a3 , A4 const& a4, A5 const& a5)
{
exporter(a0,a1,a2,a3,a4,a5);
}
};
\end{lstlisting}

On peut alors écrire dans le BOOST\_PYTHON\_MODULE :
\begin{lstlisting}
def(
       "exporter",
       py::function<
       exporter_fwd
       ,mpl::vector<
    // Exporter<Mesh<Simplex<2>>,1>
    //typename Feel::detail::compute_exporter_return<Exporter<Mesh<Simplex<2>>>>::ptrtype 
    std::shared_ptr<Exporter<Mesh<Simplex<2> >,1>> 
    , tag::mesh(Mesh<Simplex<2>>)
    , tag::fileset*(bool)
    , tag::order*(int)
    , tag::name*(std::string)
    , tag::geo*(std::string)
    , tag::path*(std::string) 
    >
    >()
    );
\end{lstlisting}

On obtient malheureusement les mêmes résultats qu'avec loadMesh, c'est à dire des erreurs sur la reconnaissance de la méthode associée

\section{Derniers Tests}

J'ai effectuer quelques tests supplémentaires sur la méthode loadMesh afin de comprendre et de peut-être corriger l'erreur obtenue.

J'ai donc tout d'abord réécrit une nouvelle BOOST\_PARAMETER\_FUNCTION avec pour seul argument le maillage mesh, définit comme suivant :

\begin{lstlisting}
 BOOST_PARAMETER_FUNCTION(
            ( typename Feel::detail::mesh<Args>::ptrtype ), // return type
            loadMesh2,    // 2. function name

            tag,           // 3. namespace of tag types

            ( required
              ( mesh, *)

            ) // 4. one required parameter, and

            )
    {
        loadMesh(_mesh=mesh);
    }
\end{lstlisting}

On obtient cependant les mêmes résultats d'erreurs suite au wrapping suivant :
\begin{lstlisting}
def(
            "loadmesh2",
            py::function<
            loadMesh2_fwd,
            mpl::vector<
            boost::shared_ptr<Mesh<Simplex<2>>>,
            tag::mesh (Mesh<Simplex<2>>*)
            >
            >()
       );
\end{lstlisting}

\vspace{0.5 cm}
Les 2 cas suivant peuvent être plus intéressant car le module associé compile mais cause des erreurs à l'exécution du script Python.

\vspace{0.5 cm}
Premièrement, on implémente une simple méthode qui à partir d'un objet de type Mesh appelle loadMesh et retourne le résultat.
\begin{lstlisting}
boost::shared_ptr<Mesh<Simplex<2>>> loadMesh3 (Mesh<Simplex<2>>* mesh)
 {
    return loadMesh(_mesh=mesh);
 }
\end{lstlisting}

\vspace{0.5 cm}
Deuxièmement on implemente une méthode reprenant l'intégralité du code présent dans la BOOST\_PARAMETER\_FUNCTION avec les valeurs par défaut sauf pour le maillage.
\begin{lstlisting}
boost::shared_ptr<Mesh<Simplex<2>>> loadMesh4 (Mesh<Simplex<2>>* mesh)
       {
#if defined(__clang__)
#pragma clang diagnostic push
#pragma clang diagnostic ignored "-Wunsequenced"
#endif
    //typedef typename Feel::detail::mesh<Args>::type _mesh_type;
    //typedef typename Feel::detail::mesh<Args>::ptrtype _mesh_ptrtype;

    // look for mesh_name in various directories (executable directory, current directory. ...)
    // return an empty string if the file is not found

    fs::path mesh_name=fs::path(Environment::findFile(option(_name="gmsh.filename").template as<std::string>() ));
    LOG_IF( WARNING, mesh_name.extension() != ".geo" && mesh_name.extension() != ".msh" )
    << "Invalid filename " <<  " it should have either the .geo or .msh extension\n";


    if ( mesh_name.extension() == ".geo" )
    {

    return createGMSHMesh(
    _mesh=mesh,
    _desc= boost::shared_ptr<gmsh_type>(),
    _h=option(_name="gmsh.hsize").template as<double>() ,
    _straighten=option(_name="gmsh.straighten").template as<bool>() ,
    _refine=option(_name="gmsh.refine").template as<int>() ,
    _update=MESH_CHECK|MESH_UPDATE_FACES|MESH_UPDATE_EDGES ,
    _physical_are_elementary_regions=option(_name="gmsh.physical_are_elementary_regions").template as<bool>() ,
    _force_rebuild=option(_name="gmsh.rebuild").template as<bool>() ,
    _worldcomm=Environment::worldComm(),
    _respect_partition=option(_name="gmsh.respect_partition").template as<bool>() ,
    _rebuild_partitions=option(_name="gmsh.partition").template as<bool>(),
    _rebuild_partitions_filename=option(_name="gmsh.filename").template as<std::string>() ,
    _partitions=Environment::worldComm().globalSize(),
    _partitioner=option(_name="gmsh.partitioner").template as<int>(),
    _partition_file=0           
    );
    }

    if ( mesh_name.extension() == ".msh"  )
    {
    return loadGMSHMesh( _mesh=mesh,
    _filename=mesh_name.string(),
    _straighten=option(_name="gmsh.straighten").template as<bool>() ,
    _refine=option(_name="gmsh.refine").template as<int>() ,
    _update=MESH_CHECK|MESH_UPDATE_FACES|MESH_UPDATE_EDGES ,
    _physical_are_elementary_regions=option(_name="gmsh.physical_are_elementary_regions").template as<bool>() ,
    _worldcomm=Environment::worldComm(),
    _respect_partition=option(_name="gmsh.respect_partition").template as<bool>() ,
    _rebuild_partitions=option(_name="gmsh.partition").template as<bool>(),
    _rebuild_partitions_filename=option(_name="gmsh.filename").template as<std::string>() ,
    _partitions=Environment::worldComm().globalSize(),
    _partitioner=option(_name="gmsh.partitioner").template as<int>(),
    _partition_file=0   
    );

    }

    LOG(WARNING) << "File " << mesh_name << " not found, generating instead an hypercube in " <<  "D geometry and mesh...";
    return createGMSHMesh(
    _mesh=mesh,
    _desc= domain( _name=option(_name="gmsh.domain.shape").template as<std::string>(), _h=option(_name="gmsh.hsize").template as<double>()  ),
    _h=option(_name="gmsh.hsize").template as<double>() ,
    _refine=option(_name="gmsh.refine").template as<int>() ,
    _update=MESH_CHECK|MESH_UPDATE_FACES|MESH_UPDATE_EDGES ,
    _physical_are_elementary_regions=option(_name="gmsh.physical_are_elementary_regions").template as<bool>() ,
    _force_rebuild=option(_name="gmsh.rebuild").template as<bool>() ,
    _worldcomm=Environment::worldComm(),
    _respect_partition=option(_name="gmsh.respect_partition").template as<bool>() ,
    _rebuild_partitions=option(_name="gmsh.partition").template as<bool>(),
    _rebuild_partitions_filename=option(_name="gmsh.filename").template as<std::string>() ,
        _partitions= Environment::worldComm().globalSize(),
        _partitioner=option(_name="gmsh.partitioner").template as<int>(),
        _partition_file=0   

            );
#if defined(__clang__)
#pragma clang diagnostic pop
#endif
} // loadMesh
\end{lstlisting}

On obtient cependant à l'exécution un erreur de type free(): invalid pointer 

\begin{lstlisting}
Error in `/usr/bin/python': free(): invalid pointer: 0x000000000172cdf0 ***
*** Aborted at 1405084752 (unix time) try "date -d @1405084752" if you are using GNU date ***
PC: @     0x7f7d584dd407 (unknown)
*** SIGABRT (@0x5ed0000fa02) received by PID 64002 (TID 0x7f7d595c7700) from PID 64002; stack trace: ***
    @     0x7f7d591978f0 (unknown)
    @     0x7f7d584dd407 (unknown)
    @     0x7f7d584e0508 (unknown)
    @     0x7f7d58516e44 (unknown)
    @     0x7f7d58520b8e (unknown)
    @     0x7f7d58521896 (unknown)
    @     0x7f7d230195d2 Feel::Mesh<>::~Mesh()
    @     0x7f7d232cddfe boost::checked_delete<>()
    @     0x7f7d232cde79 boost::detail::sp_counted_impl_p<>::dispose()
    @     0x7f7d23017392 boost::detail::sp_counted_base::release()
    @     0x7f7d2301732d boost::detail::shared_count::~shared_count()
    @     0x7f7d230172ec boost::shared_ptr<>::~shared_ptr()
    @     0x7f7d230171ee boost::python::detail::invoke<>()
    @     0x7f7d2301708f boost::python::detail::caller_arity<>::impl<>::operator()()
    @     0x7f7d23016787 boost::python::objects::caller_py_function_impl<>::operator()()
    @     0x7f7d3e9f100a (unknown)
    @     0x7f7d3e9f1378 (unknown)
    @     0x7f7d3e9fb653 (unknown)
    @     0x7f7d3e9efc83 (unknown)
    @           0x5312b5 (unknown)
    @           0x5662f0 (unknown)
    @           0x466070 (unknown)
    @           0x4665a9 (unknown)
    @           0x46720e (unknown)
    @     0x7f7d584c9b45 (unknown)
    @           0x56e67e (unknown)
Aborted
\end{lstlisting}


\section{Travaux en cours}
- Suite du travail sur la méthode loadMesh et exporter
- Étude des solutions possibles proposées ( ImportGmsh<MeshType> ,.....)
- Recherche et documentation sur la suite des classes et méthodes 

\section{Problèmes rencontrés en cours de semaine}
- Soucis de wrapping avec les BOOST\_PARAMETER\_FUNCTION pour loadMesh et exporter 
\end{document}
	