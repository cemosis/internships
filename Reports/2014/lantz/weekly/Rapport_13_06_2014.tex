\documentclass[12pt]{article}
 
\usepackage[utf8]{inputenc}
\usepackage[T1]{fontenc}
\usepackage[french]{babel}
\usepackage{enumerate}
\usepackage{listingsutf8}
\usepackage{xcolor}
\usepackage{graphicx}
\usepackage{amssymb}
\usepackage{amsmath}
\usepackage{geometry}
\usepackage{float}
\usepackage{tikz}
\usepackage{url}



\geometry {
	left = 2.5cm,
	right = 2.5cm,
	bottom = 2.5cm ,
	top = 2.5cm ,
}

\lstdefinestyle{customjava}{
 belowcaptionskip=1\baselineskip,
 breaklines=true,
 frame=single,
 %linewidth=7.5cm,
 framexleftmargin=5mm,
 %frameround=tttt,
 %framexrightmargin=5mm,
 xleftmargin=\parindent,
 language=c++,
 showstringspaces=false,
 basicstyle=\footnotesize\ttfamily,
 keywordstyle=\color{green!40!black},
 ndkeywordstyle=\color{orange},
 commentstyle=\color{purple!40!black},
 identifierstyle=\color{blue},
 stringstyle=\color{red},
 numbers=left,
 numbersep=7pt,
 inputencoding=utf8
}

\frenchbsetup{StandardLists=true}
\title {Rapport : Wrapping Python}
\author {Thomas \emph{LANTZ}}

\lstset{style=customjava, emph={int,double,void, Double}, emphstyle=\color{red}, emph={[2]wavJava, Spectrum}, emphstyle=[2]\color{orange}}

\begin{document}
\maketitle 

\section{Introduction}

Pour cette deuxième semaine, après avoir continuer mes recherches sur le wrapping Python, j'ai  pût commencer mes tentatives de wrapper une application basique du repository Feelpp.

\section{Création d'un dossier python, premiers commits et Pull Request}

Afin de bien séparer mes travaux sur le wrapping Python, j'ai créer un nouveau dossier nommé python dans feelpp/doc/manual/ \\

J'y ai ainsi déposer mes premières tentatives de wrapping sur le fichier myfunctionspace.cpp (accompagné de son fichier configuration myfunctionspace.cfg) issu du dossier feelpp/doc/manual/tutorial. \\

Il a ensuite fallu ajouter tous ces fichiers au repository de Github.Pour se faire, il suffit d'utiliser les commandes suivantes : \\

\begin{lstlisting}
git add myfunctionspace.cpp
git commit -m"first wrapping test" myfunctionspace.cpp
git push
\end{lstlisting}

Ces commandes permettent respectivement de lier, puis d'annoncer un envoi accompagné d'un commentaire pour enfin l'envoyer sur la plateforme Github.\\
Cependant, travaillant sur une copie réalisé à partir du Fork du repository de Feelpp, ceci ne sera réalisé que sur celle-ci. \\

Il faut alors réalisé une demande de Pull Request .Pour cela, à partir de l'interface du site web, il suffit alors de comparer les deux versions et choisir d'envoyer les modifications au repository "maitre".

\section{CMakeLists et premiers essais }

Afin de pouvoir wrapper l'application contenue dans le fichier myfunctionspace.cpp avec Boost Python, il faut tout d'abord commencer par y ajouter une méthode, qui va définir et créer à l'exécution le module Python associé, et qui prendra la forme suivante.\\

Code de l'application :
\begin{lstlisting}
#include <feel/feel.hpp>
#include <string>
using namespace Feel;

#include<iostream>


int test (int argc,char** argv)
{

    //Initialize Feel++ Environment
    Environment env( _argc=argc, _argv=argv,
                     _about=about( _name="myfunctionspace",
                                   _author="Feel++ Consortium",
                                   _email="feelpp-devel@feelpp.org" )  );

    //! [mesh]
    // create the mesh
    auto mesh = loadMesh(_mesh=new Mesh<Simplex<2>>);
    //! [mesh]

    //! [space]
    // function space \f$ X_h \f$ using order 2 Lagrange basis functions
    auto Xh = Pch<2>( mesh );
    //! [space]

    //! [expression]
    auto g = expr( soption(_name="functions.g"));
    auto gradg = grad<2>(g);
    //! [expression]

    //! [interpolant]
    // elements of \f$ u,w \in X_h \f$
    auto u = Xh->element( "u" );
    auto w = Xh->element( "w" );
    // build the interpolant of u
    u.on( _range=elements( mesh ), _expr=g );
    // build the interpolant of the interpolation error
    w.on( _range=elements( mesh ), _expr=idv( u )-g );

    // compute L2 norms
    double L2g = normL2( elements( mesh ), g );
    double H1g = normL2( elements( mesh ), _expr=g,_grad_expr=gradg );
    double L2uerror = normL2( elements( mesh ), ( idv( u )-g ) );
    double H1uerror = normH1( elements( mesh ), _expr=( idv( u )-g ),
                              _grad_expr=( gradv( u )-gradg ) );
    std::cout << "||u-g||_0 = " << L2uerror/L2g << std::endl;
    std::cout << "||u-g||_1 = " << H1uerror/H1g << std::endl;
     //! [interpolant]

    //! [export]
    // export for post-processing
    auto e = exporter( _mesh=mesh );
    // save interpolant
    e->add( "g", u );
    // save interpolant of interpolation error
    e->add( "u-g", w );

    e->save();
    //! [export]
  
}
//! [all]
\end{lstlisting}

Ajout du BOOST\_PYTHON\_MODULE :

\begin{lstlisting}
#include <boost/python.hpp>
using namespace boost::python;

BOOST_PYTHON_MODULE(libFunct)
{
    def("test",test); 
    def("wrap",wrap);
    def("main",main);
   
}
\end{lstlisting}

Ceci permettra alors de créer une librairie nommé libFunct qui comprendra les méthodes définit dedans, et donc ici celle qui nous intéresse : test.\\
Il faut encore cependant demander à construire cette librairie.Pour se faire, on utilisera alors un CMakeLists.txt qui sera d'ailleurs utile vu qu'on se trouve dans un nouveau dossier.
On a alors :

\begin{lstlisting}[language=make]
include_directories(${Boost_INCLUDE_DIRS}) 
add_executable(functspace myfunctionspace.cpp myfunctionspace.cfg)
add_library(Funct SHARED myfunctionspace.cpp)
target_link_libraries(Funct ${FEELPP_LIBRARIES})
target_link_libraries(functspace Funct)
target_link_libraries(functspace ${PYTHON_LIBRARIES} ${Boost_LIBRARIES})
\end{lstlisting}

Les 2 lignes suivantes sont celles permettant la création et l'utilisation la librairie voulue :\\
-add\_library(Funct SHARED myfunctionspace.cpp) permet la création du module définit dans le fichier myfunctionspace.cpp\\
-target\_link\_libraries(Funct \$\{FEELPP\_LIBRARIES\}) ajoutera,quand à elle,les librairies contenues dans Feelpp, ici Python et Boost qui nous sont utiles, à celle de Funct.\\

Il suffit alors après un cmake de faire un make dans le dossier python de /data pour obtenir la librairie libFunct. Il ne reste plus qu'à l'utiliser ainsi :

\begin{lstlisting}[language=Python]
>>> import libFunct
>>> libFunct.test(1,...)
\end{lstlisting}

On se retrouve alors face à un nouveau problème.En effet, pour appeller la méthode test, on a besoin d'un int ainsi que d'un char**, type qui n'est pas présent dans Python.\\
On a alors besoin d'ajouter une nouvelle méthode, qui à partir d'un objet Python (ici boost::python::list) va récupérer ces objets et apeller la méthode test avec eux.

\begin{lstlisting}
void wrap( boost::python::list argv)
{
    int argc = boost::python::len(argv);
    std::cout << argc << std::endl ;
    // - Could be a one-liner call to extract<>():-
    
    char** pyarg =new char* [argc];
    boost::python::stl_input_iterator<std::string> begin(argv), end;
    int i=0;
    while (begin != end)
    {
        std::cout << *begin << std::endl ;
        pyarg[i] =strdup((*begin).c_str());
        begin++;
        i++;
    }
   // pyarg[i] =new char();
   // *pyarg[i]='\0'; 
               /// ----------
    std::cout << pyarg[0] << std::endl;
    test(argc,pyarg);

    
    for(int i=0;i<argc;i++)
        delete pyarg[i];
    delete[] pyarg;
}
\end{lstlisting}

Cependant, cette méthode rencontre des problème liés à MPI lors de son exécution avec du code Feel++.En supprimant ce qui lui est lié, on obtient alors de résultats satisfaisant.\\
Par exemple :
\begin{lstlisting}
int test (int argc,char** argv)
{
    std::cout << "It's working !!!!" << std::endl;
    return 1;  
}
\end{lstlisting}

\newpage
Après exécution du script Python suivant :
\begin{lstlisting}[language=Python]
#!/usr/bin/python

import libFunct
import sys

libFunct.wrap(sys.argv)
\end{lstlisting}

On obtient :
\begin{verbatim}
1
./Test.py
./Test.py
It's working !!!!
\end{verbatim}


\section{Travaux en cours}
-Mise en application et perfectionnement de la méthode wrap permettant la récupération et l'exécution du programme principal.\\

-Inclusion de MPI (Open MPI) dans le code Python pour l'exécution du code Feel++.\\

\section{Problèmes rencontrés en cours de semaine}
-Création et modification du nouveau CMakeLists du dossier python.\\

-Récupération des données nécessaires à l'exécution du programme principal.\\

-Erreurs d'exécution MPI à partir de Python avec du code Feel++ .\\

\end{document}
