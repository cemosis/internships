\documentclass[12pt]{article}
 
\usepackage[utf8]{inputenc}
\usepackage[T1]{fontenc}
\usepackage[french]{babel}
\usepackage{enumerate}
\usepackage{listingsutf8}
\usepackage{xcolor}
\usepackage{graphicx}
\usepackage{amssymb}
\usepackage{amsmath}
\usepackage{geometry}
\usepackage{float}
\usepackage{tikz}
\usepackage{url}



\geometry {
	left = 2.5cm,
	right = 2.5cm,
	bottom = 2.5cm ,
	top = 2.5cm ,
}

\lstdefinestyle{customjava}{
 belowcaptionskip=1\baselineskip,
 breaklines=true,
 frame=single,
 %linewidth=7.5cm,
 framexleftmargin=5mm,
 %frameround=tttt,
 %framexrightmargin=5mm,
 xleftmargin=\parindent,
 language=c++,
 showstringspaces=false,
 basicstyle=\footnotesize\ttfamily,
 keywordstyle=\color{green!40!black},
 ndkeywordstyle=\color{orange},
 commentstyle=\color{purple!40!black},
 identifierstyle=\color{blue},
 stringstyle=\color{red},
 numbers=left,
 numbersep=7pt,
 inputencoding=utf8
}

\frenchbsetup{StandardLists=true}
\title {Rapport : Wrapping Python}
\author {Thomas \emph{LANTZ}}

\lstset{style=customjava, emph={int,double,void, Double}, emphstyle=\color{red}, emph={[2]wavJava, Spectrum}, emphstyle=[2]\color{orange}}

\begin{document}
\maketitle 

\section{Introduction}

Pour cette septième semaine de stage, j'ai tenté de résoudre les soucis de la semaine dernière autour du wrapping des méthodes loadMesh et exporter.
Pour se faire, j'ai utilisé diverses méthodes que je développerais pour certaines plus en détail dans la suite de ce rapport 

\section{Wrapping de la méthode loadMesh}

On revient une nouvelle fois sur la méthode loadMesh qui continue de poser quelques problèmes de wrapping. On s'était laissé sur 2 exemples qui compilait la semaine dernière, mais qui provoquait une erreur de segmentation à l'exécution .

Pour rappel, voici ces 2 méthodes :

\subsection{Méthode 1}
\begin{lstlisting}
boost::shared_ptr<Mesh<Simplex<2>>> loadMesh3 (Mesh<Simplex<2>>* mesh)
 {
    return loadMesh(_mesh=mesh);
 }
\end{lstlisting}

\vspace{0.5 cm}
\subsection{Méthode 2}
\begin{lstlisting}
boost::shared_ptr<Mesh<Simplex<2>>> loadMesh4 (Mesh<Simplex<2>>* mesh)
       {
#if defined(__clang__)
#pragma clang diagnostic push
#pragma clang diagnostic ignored "-Wunsequenced"
#endif
    //typedef typename Feel::detail::mesh<Args>::type _mesh_type;
    //typedef typename Feel::detail::mesh<Args>::ptrtype _mesh_ptrtype;

    // look for mesh_name in various directories (executable directory, current directory. ...)
    // return an empty string if the file is not found

    fs::path mesh_name=fs::path(Environment::findFile(option(_name="gmsh.filename").template as<std::string>() ));
    LOG_IF( WARNING, mesh_name.extension() != ".geo" && mesh_name.extension() != ".msh" )
    << "Invalid filename " <<  " it should have either the .geo or .msh extension\n";


    if ( mesh_name.extension() == ".geo" )
    {

    return createGMSHMesh(
    _mesh=mesh,
    _desc= boost::shared_ptr<gmsh_type>(),
    _h=option(_name="gmsh.hsize").template as<double>() ,
    _straighten=option(_name="gmsh.straighten").template as<bool>() ,
    _refine=option(_name="gmsh.refine").template as<int>() ,
    _update=MESH_CHECK|MESH_UPDATE_FACES|MESH_UPDATE_EDGES ,
    _physical_are_elementary_regions=option(_name="gmsh.physical_are_elementary_regions").template as<bool>() ,
    _force_rebuild=option(_name="gmsh.rebuild").template as<bool>() ,
    _worldcomm=Environment::worldComm(),
    _respect_partition=option(_name="gmsh.respect_partition").template as<bool>() ,
    _rebuild_partitions=option(_name="gmsh.partition").template as<bool>(),
    _rebuild_partitions_filename=option(_name="gmsh.filename").template as<std::string>() ,
    _partitions=Environment::worldComm().globalSize(),
    _partitioner=option(_name="gmsh.partitioner").template as<int>(),
    _partition_file=0           
    );
    }

    if ( mesh_name.extension() == ".msh"  )
    {
    return loadGMSHMesh( _mesh=mesh,
    _filename=mesh_name.string(),
    _straighten=option(_name="gmsh.straighten").template as<bool>() ,
    _refine=option(_name="gmsh.refine").template as<int>() ,
    _update=MESH_CHECK|MESH_UPDATE_FACES|MESH_UPDATE_EDGES ,
    _physical_are_elementary_regions=option(_name="gmsh.physical_are_elementary_regions").template as<bool>() ,
    _worldcomm=Environment::worldComm(),
    _respect_partition=option(_name="gmsh.respect_partition").template as<bool>() ,
    _rebuild_partitions=option(_name="gmsh.partition").template as<bool>(),
    _rebuild_partitions_filename=option(_name="gmsh.filename").template as<std::string>() ,
    _partitions=Environment::worldComm().globalSize(),
    _partitioner=option(_name="gmsh.partitioner").template as<int>(),
    _partition_file=0   
    );

    }

    LOG(WARNING) << "File " << mesh_name << " not found, generating instead an hypercube in " <<  "D geometry and mesh...";
    return createGMSHMesh(
    _mesh=mesh,
    _desc= domain( _name=option(_name="gmsh.domain.shape").template as<std::string>(), _h=option(_name="gmsh.hsize").template as<double>()  ),
    _h=option(_name="gmsh.hsize").template as<double>() ,
    _refine=option(_name="gmsh.refine").template as<int>() ,
    _update=MESH_CHECK|MESH_UPDATE_FACES|MESH_UPDATE_EDGES ,
    _physical_are_elementary_regions=option(_name="gmsh.physical_are_elementary_regions").template as<bool>() ,
    _force_rebuild=option(_name="gmsh.rebuild").template as<bool>() ,
    _worldcomm=Environment::worldComm(),
    _respect_partition=option(_name="gmsh.respect_partition").template as<bool>() ,
    _rebuild_partitions=option(_name="gmsh.partition").template as<bool>(),
    _rebuild_partitions_filename=option(_name="gmsh.filename").template as<std::string>() ,
        _partitions= Environment::worldComm().globalSize(),
        _partitioner=option(_name="gmsh.partitioner").template as<int>(),
        _partition_file=0   

            );
#if defined(__clang__)
#pragma clang diagnostic pop
#endif
} // loadMesh
\end{lstlisting}
\vspace{0.5 cm}

Après installation et utilisation d'un gdb pour Python permettant de retrouver les causes d'une erreur de segmentation ainsi que quelques tests d'affichages, j'ai pu m'apercevoir que l'erreur était causé à la destruction de l'objet construit par loadMesh ,à la fin du script.
On a donc décider de laisser cela tel quel pour le moment.

\section{Wrapping de la méthode exporter }

Suite aux problèmes précédents rencontrés la semaine dernière, semblables à ceux de loadMesh, je me suis dirigé vers une des solutions proposées : wrapper la classe Exporter ainsi que les méthodes nécessaires à son utilisation afin de reproduire le fonctionnement de la méthode exporter.
\vspace{0.5 cm}

On a donc besoin de wrapper la classe Exporter<MeshType,N> ainsi que la méthode New permettant l'initialisation d'un de ces objets. Il nous faut également la méthode setMesh, addRegions et save .
\vspace{0.5 cm}

Cependant, la classe Exporter possédant des méthodes virtual dont save , on ne peut pas wrapper simplement la classe telle quelle. On va donc choisir un exportateur héritant de cette classe, ici ExporterEnsightGold.

\section{Wrapping de ExporterEnsightGold}

On va donc commencer par wrapper la classe ExporterEnsightGold  ainsi que les méthodes explicitées auparavant , ce qui nous donne alors :

\begin{lstlisting}
class_<ExporterEnsightGold<Mesh<Simplex<2>>,1>>("Exporter",init<WorldComm>())
        .def("new",&Exporter<Mesh<Simplex<2>>>::New1)
        .def("setMesh",&Exporter<Mesh<Simplex<2>>>::setMesh) 
        .def("addRegions",&Exporter<Mesh<Simplex<2>>>::addRegions)
        .def("save",&ExporterEnsightGold<Mesh<Simplex<2>>,1>::save);
\end{lstlisting}
\vspace{0.5 cm}
On utilisera les méthodes setMesh,addRegions et New présent dans la classe Exporter et save qui est définit dans ExporterEnsightGold.
\vspace{0.5 cm}

Cependant, la création d'une instance de ExporterEnsightGold nécessite l'utilisation de la méthode New,qui renvoie un objet de type boost$::$shared\_ptr<Exporter<MeshType,N$>>$
Or comme évoqué précédemment, on ne peut pas instancier un objet de la classe Exporter<MeshType,N> et donc on ne peut également pas le wrapper.
\vspace{0.5 cm}

Pour contourner ce problème, j'ai écrit une nouvelle méthode qui à partir d'un boost::shared\_ptr sur un maillage ( ce qu'on obtient à partir de loadMesh ) va créer l'exportateur et y appliquer les méthodes nécessaires directement dessus.\\
De ce fait, Python n'aura pas besoin de gérer l'objet Exporter pour lui appliquer les méthodes dans le script.

La méthode peut s'écrire : 

\begin{lstlisting}
template<typename MeshType,int N>
void expo2 ( boost::shared_ptr<MeshType> m)
{
  auto x=Exporter<MeshType,N>::New();
  x->setMesh(m);
  x->addRegions();
  x->save();
}
\end{lstlisting}

et la version wrappée donne :

\begin{lstlisting}
def("export",expo2<Mesh<Simplex<2>>,1>);
\end{lstlisting}


\section{Premiers résultats}

Maintenant qu'on a terminé de récupérer l'ensemble des méthodes et classes nécessaires à la reproduction de mymesh.cpp dans un module Python , on va pouvoir l'utiliser dans un script destiné à cet effet.

Tout d'abord , la création du module complet :

\begin{lstlisting}
#include <feel/feel.hpp>
#include <boost/python.hpp>
#include <boost/python/stl_iterator.hpp>
#include <mpi4py/mpi4py.h>

#include <boost/shared_ptr.hpp>
#include <boost/parameter/keyword.hpp>
#include <boost/parameter/preprocessor.hpp>
#include <boost/parameter/binding.hpp>
#include <boost/parameter/python.hpp>
#include <boost/python.hpp>
#include <boost/mpl/vector.hpp>

#include<feel/feelcore/environment.hpp>
#include<feel/feelfilters/loadmesh.hpp>
#include<feel/feelfilters/exporter.hpp>
#include<feel/feelfilters/detail/mesh.hpp>


using namespace boost::python;
using namespace Feel;

namespace py = boost::parameter::python;



template<typename MeshType,int N>
void expo2 ( boost::shared_ptr<MeshType> m)
{
  auto x=Exporter<MeshType,N>::New();
  x->setMesh(m);
  x->addRegions();
  x->save();
}


BOOST_PYTHON_MODULE(libPyFeelpp)
{
   
   if (import_mpi4py()<0) return ;

 class_<Feel::detail::Environment,boost::noncopyable>("Environment", init<boost::python::list>()) 
       .def("worldComm",&Feel::detail::Environment::worldComm,return_value_policy<copy_non_const_reference>())
        .staticmethod("worldComm");



class_<Feel::Simplex<2>>("Simplex",init<>())
        .def("dim",&Feel::Simplex<2>::dimension);

class_<Feel::Hypercube<2>>("Hypercube",init<>())
        .def("dim",&Feel::Hypercube<2>::dimension);
        


class_<Feel::Mesh<Feel::Simplex<2>>,boost::shared_ptr<Feel::Mesh<Feel::Simplex<2>>>,boost::noncopyable>("Mesh",init<>())
        .def("new",&Feel::Mesh<Simplex<2>>::New)
        .staticmethod("new")
        .def("clear",&Feel::Mesh<Simplex<2>>::clear);
  
 class_<WorldComm>("WorldComm",init<>());

 class_<ExporterEnsightGold<Mesh<Simplex<2>>,1>>("Exporter",init<WorldComm>())
        .def("setMesh",&Exporter<Mesh<Simplex<2>>>::setMesh) 
        .def("addRegions",&Exporter<Mesh<Simplex<2>>>::addRegions)
        .def("save",&ExporterEnsightGold<Mesh<Simplex<2>>,1>::save);
      
        
    
    def("loadMesh",loadMesh3);
    def("export",expo2<Mesh<Simplex<2>>,1>);
    
}
\end{lstlisting}

On écrit alors le script Python de la façon suivante :

\begin{lstlisting}
#!/usr/bin/python

from mpi4py import MPI
import libPyFeelpp
import sys

z=libPyFeelpp.Environment(sys.argv)

s=libPyFeelpp.Simplex()
m=libPyFeelpp.Mesh.new()

l=libPyFeelpp.loadMesh(m)

w=libPyFeelpp.Environment.worldComm()

x=libPyFeelpp.export(l);

\end{lstlisting}

On peut alors retrouver les résultats dans $/data/"name"/feel/EnvTest.py/np\_1$ et on retrouve alors le maillage obtenu par exécution du programme mymesh.cpp.

\includegraphics[scale=1]{Capture d’écran 2014-07-18 à 17.33.53.png} 

\section{Travaux en cours}
- Début du wrapping du programme ls\_laplacian dans $feelpp/quickstart$\\

- Création d'une option dans Feel++ permettant de choisir ou non la compilation du module Python

\section{Problèmes rencontrés en cours de semaine}
- Soucis de wrapping avec les BOOST\_PARAMETER\_FUNCTION pour loadMesh et exporter \\

- Problèmes avec la classe Exporter qui ne peut être instancié à cause de ses méthodes virtua
\end{document}
	