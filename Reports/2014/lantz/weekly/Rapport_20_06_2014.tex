\documentclass[12pt]{article}
 
\usepackage[utf8]{inputenc}
\usepackage[T1]{fontenc}
\usepackage[french]{babel}
\usepackage{enumerate}
\usepackage{listingsutf8}
\usepackage{xcolor}
\usepackage{graphicx}
\usepackage{amssymb}
\usepackage{amsmath}
\usepackage{geometry}
\usepackage{float}
\usepackage{tikz}
\usepackage{url}



\geometry {
	left = 2.5cm,
	right = 2.5cm,
	bottom = 2.5cm ,
	top = 2.5cm ,
}

\lstdefinestyle{customjava}{
 belowcaptionskip=1\baselineskip,
 breaklines=true,
 frame=single,
 %linewidth=7.5cm,
 framexleftmargin=5mm,
 %frameround=tttt,
 %framexrightmargin=5mm,
 xleftmargin=\parindent,
 language=c++,
 showstringspaces=false,
 basicstyle=\footnotesize\ttfamily,
 keywordstyle=\color{green!40!black},
 ndkeywordstyle=\color{orange},
 commentstyle=\color{purple!40!black},
 identifierstyle=\color{blue},
 stringstyle=\color{red},
 numbers=left,
 numbersep=7pt,
 inputencoding=utf8
}

\frenchbsetup{StandardLists=true}
\title {Rapport : Wrapping Python}
\author {Thomas \emph{LANTZ}}

\lstset{style=customjava, emph={int,double,void, Double}, emphstyle=\color{red}, emph={[2]wavJava, Spectrum}, emphstyle=[2]\color{orange}}

\begin{document}
\maketitle 

\section{Introduction}

Durant cette troisième semaine de stage, j'ai continué et achever le wrapping de l'application de base myfunctionspace, avant de pouvoir m'attaquer à un wrapping plus ardu ,celui de la classe Environment , utilisée dans l'intégralité des programmes Feel++.

\section{Suite et premier résultat du wrapping de Myfunctionspace }

La semaine dernière s'était achevée sur les premiers résultats du wrapping, en ayant supprimer toute trace de code appartenant à la librairie Feelpp. Pour rappel, on avait alors la méthode wrap qui appelait la méthode test, contenant le code voulu, à partir d'une liste Python.
 
 Le BOOST\_PYTHON\_MODULE s'écrivait alors :
 
\begin{lstlisting}
#include <boost/python.hpp>
using namespace boost::python;

BOOST_PYTHON_MODULE(libFunct)
{
    def("test",test); 
    def("wrap",wrap);
    def("main",main);
   
}
\end{lstlisting}

et par l'exécution du script Python suivant :

\begin{lstlisting}[language=Python]
#!/usr/bin/python

import libFunct
import sys

libFunct.wrap(sys.argv)
\end{lstlisting}

On obtenait  :
\begin{verbatim}
1
./Test.py
./Test.py
It's working !!!!
\end{verbatim}

On veut maintenant faire la même chose mais avec le code Feel++ de myfunctionspace.
Pour se faire, il nous faut intégrer et initialiser MPI dans le module Python ainsi que dans le script Python.\\

On utilisera pour se faire la librairie mpi4py qui permet d'initialiser et donc d'utiliser MPI en Python.\\

Ainsi la définition du module devient :
\begin{lstlisting}
#include <mpi4py/mpi4py.h>
#include <boost/python.hpp>
using namespace boost::python;

// build the module, named libFunct, from previous methods
BOOST_PYTHON_MODULE(libFunct)
{
    if (import_mpi4py() <0) return ;
    def("test",test); 
    def("wrap",wrap);
    def("main",main);
   
}
\end{lstlisting}

Les méthodes test et wrap deviennent  :
\begin{lstlisting}
#include <feel/feel.hpp>
#include <string>
using namespace Feel;
#include <mpi.h>
#include <iostream>


int test (int argc,char** argv)
{
    for(int i=0;i<argc;i++)
         std::cout << argv[i] << std::endl;
    //Initialize Feel++ Environment
    Environment env( _argc=argc, _argv=argv,
                     _about=about( _name="myfunctionspace",
                                   _author="Feel++ Consortium",
                                   _email="feelpp-devel@feelpp.org" )  );
    
    //! [mesh]
    // create the mesh
    auto mesh = loadMesh(_mesh=new Mesh<Simplex<2>>);
    //! [mesh]

    //! [space]
    // function space \f$ X_h \f$ using order 2 Lagrange basis functions
    auto Xh = Pch<2>( mesh );
    //! [space]

    //! [expression]
    auto g = expr( soption(_name="functions.g"));
    auto gradg = grad<2>(g);
    //! [expression]

    //! [interpolant]
    // elements of \f$ u,w \in X_h \f$
    auto u = Xh->element( "u" );
    auto w = Xh->element( "w" );
    // build the interpolant of u
    u.on( _range=elements( mesh ), _expr=g );
    // build the interpolant of the interpolation error
    w.on( _range=elements( mesh ), _expr=idv( u )-g );

    // compute L2 norms
    double L2g = normL2( elements( mesh ), g );
    double H1g = normL2( elements( mesh ), _expr=g,_grad_expr=gradg );
    double L2uerror = normL2( elements( mesh ), ( idv( u )-g ) );
    double H1uerror = normH1( elements( mesh ), _expr=( idv( u )-g ),
                              _grad_expr=( gradv( u )-gradg ) );
    std::cout << "||u-g||_0 = " << L2uerror/L2g << std::endl;
    std::cout << "||u-g||_1 = " << H1uerror/H1g << std::endl;
     //! [interpolant]

    //! [export]
    // export for post-processing
    auto e = exporter( _mesh=mesh );
    // save interpolant
    e->add( "g", u );
    // save interpolant of interpolation error
    e->add( "u-g", w );

    e->save();
    //! [export]

            return 1;  
}
//! [all]
\end{lstlisting}
\begin{lstlisting}
#include <boost/python.hpp>
#include <boost/python/stl_iterator.hpp>
#include <mpi4py/mpi4py.h>

// build an arguments recuperator from a Python object list to call the test method
void wrap( boost::python::list argv)
{
    int argc = boost::python::len(argv);
    std::cout << argc << std::endl ;
        
    char** pyarg =new char* [argc+1];
    boost::python::stl_input_iterator<std::string> begin(argv), end;
    int i=0;
    while (begin != end)
    {
        std::cout << *begin << std::endl ;
        pyarg[i] =strdup((*begin).c_str());
        begin++;
        i++;
    }
    pyarg[argc]=NULL;
    test(argc,pyarg);
    std::cout << "It's working !!!!" << std::endl; 
    for(int i=0;i<argc;i++)
        delete pyarg[i];
    delete[] pyarg;
}
\end{lstlisting}
et enfin le script Python :

\begin{lstlisting}
#!/usr/bin/python

from mpi4py import MPI
import libFunct
import sys

 libFunct.wrap(sys.argv)
\end{lstlisting}

On obtient finalement à l'exécution :

\begin{verbatim}
./Test.py

1
./Test.py
./Test.py
||u-g||_0 = -nan
||u-g||_1 = -nan
It's working !!!!
\end{verbatim}

La fonction g étant défini de base dans le fichier de configuration, myfunctionspace.cfg, si l'on ne lui donne pas une valeur, on obtient des résultats comme ci-dessus.\\
On peut alors utiliser l'option functions.g contenu dans le code Feel++ de base pour lui donner une valeur. On obtient alors :

\begin{verbatim}
./Test.py --functions.g="sin(pi*x)*cos(pi*y):x:y"

2
./Test.py
--functions.g=sin(pi*x)*cos(pi*y):x:y
./Test.py
--functions.g=sin(pi*x)*cos(pi*y):x:y
||u-g||_0 = 0.000400902
||u-g||_1 = 0.0132609
It's working !!!!
\end{verbatim}

On peut alors comparer avec l'exécutable functspace créé à partir du code de base :

\begin{verbatim}
/functspace --functions.g="sin(pi*x)*cos(pi*y):x:y"

||u-g||_0 = 0.000400902
||u-g||_1 = 0.0132609
\end{verbatim}

On obtient finalement les mêmes résultats.

De même, on peut lancer l'exécution du script Python en parallèle avec la commande mpirun pour obtenir :

\begin{verbatim}
 mpirun -np 4 python Test.py --functions.g="sin(pi*x)*cos(pi*y):x:y"

2
Test.py
--functions.g=sin(pi*x)*cos(pi*y):x:y
Test.py
--functions.g=sin(pi*x)*cos(pi*y):x:y
2
Test.py
--functions.g=sin(pi*x)*cos(pi*y):x:y
Test.py
--functions.g=sin(pi*x)*cos(pi*y):x:y
2
Test.py
--functions.g=sin(pi*x)*cos(pi*y):x:y
Test.py
--functions.g=sin(pi*x)*cos(pi*y):x:y
2
Test.py
--functions.g=sin(pi*x)*cos(pi*y):x:y
Test.py
--functions.g=sin(pi*x)*cos(pi*y):x:y
||u-g||_0 = 0.000400902
||u-g||_1 = 0.0132609
||u-g||_0 = 0.000400902
||u-g||_1 = 0.0132609
||u-g||_0 = 0.000400902
||u-g||_1 = 0.0132609
||u-g||_0 = 0.000400902
||u-g||_1 = 0.0132609
It's working !!!!
It's working !!!!
It's working !!!!
It's working !!!!
\end{verbatim}

\section{Classe Environment}

On aimerait s'attaquer maintenant aux wrapping des différentes classes et méthodes de Feel++ afin de pouvoir lancer par exemple chaque ligne de myfunctionspace séparément en Python.\\

Pour débuter cela, on va commencer par s'attaquer à la classe présente dans chaque fichier : la classe Environment, utilisé comme suivant :

\begin{lstlisting}
Environment env( _argc=argc, _argv=argv,
                     _about=about( _name="myfunctionspace",
                                   _author="Feel++ Consortium",
                                   _email="feelpp-devel@feelpp.org" )  );

\end{lstlisting}

On travaillera alors dans les fichiers environment.cpp et environment.hpp de ~/feelpp/feel/feelcore/ et dans un premeir temps, plus précisement à la partie suivante : \\
\\
environment.hpp
\begin{lstlisting}
class Environment : public detail::Environment
{
public:
    BOOST_PARAMETER_CONSTRUCTOR(
        Environment, (detail::Environment), tag,
        (required
         (argc,*)
         (argv,*))
        (optional
         (desc,*)
         (desc_lib,*)
         (about,*)
         (directory,( std::string ))
            )) // no semicolon
     
      
};
\end{lstlisting}

environment.cpp
\begin{lstlisting}
Environment::Environment( int& argc, char**& argv )
    :
    M_env( argc, argv, false )
{
    S_scratchdir = scratchdir()/fs::path(argv[0]).filename();
    if ( !fs::exists( S_scratchdir ) )
        fs::create_directories( S_scratchdir );
    FLAGS_log_dir=S_scratchdir.string();

    google::AllowCommandLineReparsing();
    google::ParseCommandLineFlags(&argc, &argv, false);

    // Initialize Google's logging library.
    if ( !google::glog_internal_namespace_::IsGoogleLoggingInitialized() )
    {
        if ( argc > 0 )
            google::InitGoogleLogging(argv[0]);
        else
            google::InitGoogleLogging("feel++");
    }
    google::InstallFailureSignalHandler();

#if defined( FEELPP_HAS_TBB )
    int n = tbb::task_scheduler_init::default_num_threads();
    //int n = 2;
    //VLOG(2) << "[Feel++] TBB running with " << n << " threads\n";
    //tbb::task_scheduler_init init(2);
#endif

    S_worldcomm = worldcomm_type::New();
    CHECK( S_worldcomm ) << "Feel++ Environment: creang worldcomm failed!";
    S_worldcommSeq.reset( new WorldComm(S_worldcomm->subWorldCommSeq()) );

#if defined ( FEELPP_HAS_PETSC_H )
    PetscTruth is_petsc_initialized;
    PetscInitialized( &is_petsc_initialized );

    if ( !is_petsc_initialized )
    {
        i_initialized = true;
#if defined( FEELPP_HAS_SLEPC )
        int ierr = SlepcInitialize( &argc,&argv, PETSC_NULL, PETSC_NULL );
#else
        int ierr = PetscInitialize( &argc, &argv, PETSC_NULL, PETSC_NULL );
#endif
        boost::ignore_unused_variable_warning( ierr );
        CHKERRABORT( *S_worldcomm,ierr );
    }

    // make sure that petsc do not catch signals and hence do not print long
    //and often unuseful messages
    PetscPopSignalHandler();
#endif // FEELPP_HAS_PETSC_H

}
\end{lstlisting}

On va rencontrer le même problème que lors du wrapping de myfunctionspace avec la récupération des arguments. On tente alors d'écrire un constructeur qui prend une boost::python::list en argument :

\begin{lstlisting}
Environment::Environment(boost::python::list arg)
    {
    int argc = boost::python::len(arg);
    std::cout << argc << std::endl ;
        
    char** argv =new char* [argc+1];
    boost::python::stl_input_iterator<std::string> begin(arg), end;
    int i=0;
    while (begin != end)
    {
        std::cout << *begin << std::endl ;
        argv[i] =strdup((*begin).c_str());
        begin++;
        i++;
    }
    argv[argc]=NULL;
    
    S_scratchdir = scratchdir()/fs::path(argv[0]).filename();
    if ( !fs::exists( S_scratchdir ) )
        fs::create_directories( S_scratchdir );
    FLAGS_log_dir=S_scratchdir.string();

    google::AllowCommandLineReparsing();
    google::ParseCommandLineFlags(&argc, &argv, false);

    // Initialize Google's logging library.
    if ( !google::glog_internal_namespace_::IsGoogleLoggingInitialized() )
    {
        if ( argc > 0 )
            google::InitGoogleLogging(argv[0]);
        else
            google::InitGoogleLogging("feel++");
    }
    google::InstallFailureSignalHandler();

#if defined( FEELPP_HAS_TBB )
    int n = tbb::task_scheduler_init::default_num_threads();
    //int n = 2;
    //VLOG(2) << "[Feel++] TBB running with " << n << " threads\n";
    //tbb::task_scheduler_init init(2);
#endif

    S_worldcomm = worldcomm_type::New();
    CHECK( S_worldcomm ) << "Feel++ Environment: creang worldcomm failed!";
    S_worldcommSeq.reset( new WorldComm(S_worldcomm->subWorldCommSeq()) );

    #if defined ( FEELPP_HAS_PETSC_H )
    PetscTruth is_petsc_initialized;
    PetscInitialized( &is_petsc_initialized );

    if ( !is_petsc_initialized )
    {
        i_initialized = true;
#if defined( FEELPP_HAS_SLEPC )
        int ierr = SlepcInitialize( &argc,&argv, PETSC_NULL, PETSC_NULL );
#else
        int ierr = PetscInitialize( &argc, &argv, PETSC_NULL, PETSC_NULL );
#endif
        boost::ignore_unused_variable_warning( ierr );
        CHKERRABORT( *S_worldcomm,ierr );
    }

    // make sure that petsc do not catch signals and hence do not print long
    //and often unuseful messages
    PetscPopSignalHandler();
#endif // FEELPP_HAS_PETSC_H
     

   // Environment::Environment(argc,argv);
        for(int i=0;i<argc;i++)
        delete argv[i];
    delete [] argv ;    
}
\end{lstlisting}

On construit alors le module dans lib.cpp comme suivant :

\begin{lstlisting}
#include "environment.hpp"
#include <boost/python.hpp>
#include <boost/python/stl_iterator.hpp>
#include <mpi4py/mpi4py.h>




#include <boost/python.hpp>
using namespace boost::python;

// build the module, named libFunct, from previous methods
BOOST_PYTHON_MODULE(libEnvironment)
{
    if (import_mpi4py() <0) return ;
  class_<Feel::Environment,boost::noncopyable, bases<Feel::detail::Environment> > ("Environment",init<list>())
        ;
       
     
}
\end{lstlisting}

Cependant, pour l'instant, la création de ce module n'est pas encore fonctionnel.


\section{Travaux en cours}
- Suite du travail de wrapping sur la classe Environment.

- Recherche et documentation pour la correction des erreurs rencontrées sur ce problème.

\section{Problèmes rencontrés en cours de semaine}
- Exécution du module libFunct de myfunctionspace en parallèle .

- Erreurs de compilation pour la création du module Python de Environment .

\end{document}
