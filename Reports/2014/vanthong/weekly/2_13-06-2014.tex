\documentclass[12pt]{article}
 
\usepackage[utf8]{inputenc}
\usepackage[T1]{fontenc}
\usepackage[francais]{babel}
\usepackage{enumerate}
\usepackage{listings}
\usepackage{xcolor}
\usepackage{graphicx}
\usepackage{amssymb}
\usepackage{amsmath}
\usepackage{geometry}
\usepackage{float}
\usepackage{tikz}

\geometry {
	left = 2.5cm,
	right = 2.5cm,
	bottom = 2.5cm ,
	top = 2.5cm ,
}

\lstdefinestyle{customjava}{
 belowcaptionskip=1\baselineskip,
 breaklines=true,
 frame=single,
 %linewidth=7.5cm,
 framexleftmargin=5mm,
 %frameround=tttt,
 %framexrightmargin=5mm,
 xleftmargin=\parindent,
 language=C++,
 showstringspaces=false,
 basicstyle=\footnotesize\ttfamily,
 keywordstyle=\color{green!40!black},
 ndkeywordstyle=\color{orange},
 commentstyle=\color{purple!40!black},
 identifierstyle=\color{blue},
 stringstyle=\color{red},
 numbers=left,
 numbersep=7pt,
}
%06/06/14
\frenchbsetup{StandardLists=true}
\title {Rapport : Format de Fichiers}
\author {Benjamin \emph{VANTHONG}}

\lstset{style=customjava, emph={int,double,void, Double}, emphstyle=\color{red}, emph={[2]wavJava, Spectrum}, emphstyle=[2]\color{orange}}
\begin{document}
\maketitle 
\section {Format de Fichier HDF5}
J'ai débuté la semaine par lire la documentation sur le format de fichier HDF5 et en faisant les tutoriels proposés par le site officiel du \textbf{HDF Group}. 
Par conséquent j'ai dû installer sur ma machine les librairies et et quelques outils de visualisation de fichiers HDF5 comme \emph{HDFview} pour explorer et analyser la structure d'un fichier.\newline
Pour compiler sur ma machine, j'utilise les commandes \emph{h5cc} pour les fichiers \emph{.c} et \emph{h5++} pour les fichiers \emph{.cpp}.\newline 
Pourtant sur irma, il est préférable d'utiliser un fichier \emph{CMakeLists.txt} pour la compilation :
\begin{lstlisting}
cmake_minimum_required(VERSION 2.8)
find_package(HDF5 REQUIRED)
if (HDF5_FOUND)
    message (STATUS "HDF5 - Headers ${HDF5_INCLUDE_DIRS}")
    message (STATUS "HDF5 - Libraries ${HDF5_LIBRARIES}")
include_directories (${HDF5_INCLUDE_DIRS})
    endif ()
    find_package(MPI REQUIRED)
if (MPI_FOUND)    
    message (STATUS "MPI - Headers ${MPI_INCLUDE_DIRS}")
    message (STATUS "MPI - Libraries ${MPI_LIBRARIES}")
include_directories (${MPI_INCLUDE_PATH})
endif ()

add_executable (exemple exemple.c)
target_link_libraries (exemple ${MPI_LIBRARIES} ${HDF5_LIBRARIES})
\end{lstlisting}
Un fichier HDF5 est composé de deux types d'objets :
\begin{enumerate}
\item \textbf{HDF5 group} : un groupe de structure contenant zéro ou plusieurs objet HDF5 et des métadonnées
\item \textbf{HDF5 dataset} : un tableau de données multidimensionnel et des métadonnées
\item \textbf{HDF5 attribut} : information qui décrit l'objet HDF5
\end{enumerate}
Une métadonnée est une donnée servant à définir ou décrire une autre donnée.
\section {Manipulation GitHub}
\begin{enumerate}
\item création d'un dépôt pour les rapports hebdomadaires
\item fusion du dépôt fork et le dépôt principal pour récupérer mon nouveau répertoire de travail (~/feelpp/applications/mesh/)
\item premier commit éffectué 
\end{enumerate}
\section {Partitionnement d'un maillage}
J'ai ajouté une option dans le \emph{~/feelpp/applications/mesh/mesh.cpp} pour réccupérer le nombre de partition voulu par l'utilisateur.
\begin{lstlisting}
int main( int argc, char** argv )
{
    po::options_description opts ( "Mesh basic information and partition");
    opts.add_options()
        ( "numPartition", po::value<int>()->default_value(1), "Number of partitions" );

    // initialize Feel++ Environment
    Environment env( _argc=argc, _argv=argv,
            _desc=opts,
            _about=about( _name="mesh" ,
                _author="Feel++ Consortium",
                _email="feelpp-devel@feelpp.org" ) );

    // create a mesh with GMSH using Feel++ geometry tool
    auto numPartition = ioption(_name="numPartition");
    auto mesh = loadMesh(_mesh=new  Mesh<Simplex<FEELPP_DIM>>, _partitions=numPartition);

    if ( Environment::isMasterRank() )
    {
        std::cout << " - mesh entities" << std::endl;
        std::cout << "   number of elements : " << mesh->numGlobalElements() << std::endl;
        std::cout << "      number of faces : " << mesh->numGlobalFaces() << std::endl;
        if ( FEELPP_DIM > 2 )
            std::cout << "      number of edges : " << mesh->numGlobalEdges() << std::endl;  
        std::cout << "      number of points : " << mesh->numGlobalPoints() << std::endl;
        std::cout << "    number of vertices : " << mesh->numGlobalVertices() << std::endl;
        std::cout << " - mesh sizes" << std::endl;
        std::cout << "                h max : " << mesh->hMax() << std::endl;
        std::cout << "                h min : " << mesh->hMin() << std::endl;
        std::cout << "                h avg : " << mesh->hAverage() << std::endl;
        std::cout << "              measure : " << mesh->measure() << std::endl;

        std::cout << "Number of Partitions : " << numPartition << std::endl ;
    }


    // export results for post processing
    auto e = exporter( _mesh=mesh);
    e->addRegions();
    e->save();
}   // main
\end{lstlisting}
Durant cette semaine, j'ai passé la plupart du temps à regarder et essayer de comprendre les classes dans le répertoire ~/feelpp/feel/feelfilters/ (loadmesh.hpp, gmsh.hpp, exporter.hpp, exporter.cpp, exporter\_impl.cpp...)
\section {Travail à faire}
\begin{enumerate}
\item Continuer à faire les tutoriels sur HDF5
\item Etudier comment paraview gère les données sous format HDF5
\item Se Documenter sur XMDF
\item Apprendre à utiliser \emph{screen} pour limiter le nombre de connexion \emph{ssh} 

\end{enumerate}
\end{document}
