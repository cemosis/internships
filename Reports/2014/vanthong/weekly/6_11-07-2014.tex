\documentclass[12pt]{article}
 
\usepackage[utf8]{inputenc}
\usepackage[T1]{fontenc}
\usepackage[francais]{babel}
\usepackage{enumerate}
\usepackage{listings}
\usepackage{xcolor}
\usepackage{graphicx}
\usepackage{amssymb}
\usepackage{amsmath}
\usepackage{geometry}
\usepackage{float}
\usepackage{tikz}

\geometry {
	left = 2.5cm,
	right = 2.5cm,
	bottom = 2.5cm ,
	top = 2.5cm ,
}

\lstdefinestyle{customjava}{
 belowcaptionskip=1\baselineskip,
 breaklines=true,
 frame=single,
 %linewidth=7.5cm,
 framexleftmargin=5mm,
 %frameround=tttt,
 %framexrightmargin=5mm,
 xleftmargin=\parindent,
 language=C++,
 showstringspaces=false,
 basicstyle=\footnotesize\ttfamily,
 keywordstyle=\color{green!40!black},
 ndkeywordstyle=\color{orange},
 commentstyle=\color{purple!40!black},
 identifierstyle=\color{blue},
 stringstyle=\color{red},
 %numbers=left,
 %numbersep=7pt,
}
%06/06/14
\frenchbsetup{StandardLists=true}
\title {Rapport : Format de Fichiers}
\author {Benjamin \emph{VANTHONG}}

\lstset{style=customjava, emph={int,double,void, Double}, emphstyle=\color{red}, emph={[2]wavJava, Spectrum}, emphstyle=[2]\color{orange}}
\begin{document}
\maketitle 
\section {Sélection du type des éléments du maillage}
En fonction de la dimension et l'ordre géométrique du maillage, on sélectionne le type des éléments  :
\begin{lstlisting}
if ( mesh_type::nDim == 1 )
if ( mesh_type::Shape == SHAPE_LINE )
    M_element_type = ( mesh_type::nOrder == 1 )?"Polyline":"Edge_3";

if ( mesh_type::nDim == 2 )
{
    if ( mesh_type::Shape == SHAPE_TRIANGLE )
        M_element_type = ( mesh_type::nOrder == 1 )?"Triangle":"Tri_6";

    else if ( mesh_type::Shape == SHAPE_QUAD )
        M_element_type = ( mesh_type::nOrder == 1 )?"Quadrilateral":"Quad_8";
}

if ( mesh_type::nDim == 3 )
{
    if ( mesh_type::Shape == SHAPE_TETRA )
        M_element_type = ( mesh_type::nOrder == 1 )?"Tetrahedron":"Tet_10";

    else if ( mesh_type::Shape == SHAPE_HEXA )
        M_element_type = ( mesh_type::nOrder == 1 )?"Hexahedron":"Hex_20";
}
\end{lstlisting}
(Note j'ai fait des tests d'exportation que sur des maillages 2D et 3D, mais je pense que le 1D devrait fonctionner aussi)

\section {Correction de l'exportation des éléments}
Après plusieurs tests sur des maillages 2D et 3D, on s'est rendu compte que l'exportation des maillages avec marqueurs (physical name) ne fonctionnait pas totalement. Pour corriger cette erreur, il a fallu que je parcours l'ensemble des éléments de chaque partition pour avoir la totalité des éléments.

\begin{lstlisting}
template <typename MeshType, int N>
void Exporterhdf5<MeshType, N>::writeElements1 ()
{
    //Initialisation of partitions
    typename mesh_type::parts_const_iterator_type p_it = M_meshOut->beginParts();
    typename mesh_type::parts_const_iterator_type p_en = M_meshOut->endParts();
    M_numParts = std::distance (p_it, p_en) ;

    //Count the total number of elements
    M_maxNumElements = 0 ;
    for (int i = 0 ; i < M_numParts ; i++, p_it ++) 
    {
        auto elt_it = M_meshOut->beginElementWithMarker (p_it->first) ;
        auto elt_en = M_meshOut->endElementWithMarker (p_it->first) ;
        M_maxNumElements += std::distance (elt_it, elt_en) ;
        std::cout << "std::distance (elt_it, elt_en) : " << std::distance(elt_it, elt_en) << std::endl ;
    }

    //Number of nodes per element
    M_elementNodes = M_meshOut-> numLocalVertices () ;

    hsize_t currentSpacesDims [2] ;
    hsize_t currentSpacesDims2 [2] ;

    currentSpacesDims [0] = M_maxNumElements ;
    currentSpacesDims [1] = M_elementNodes ;

    currentSpacesDims2 [0] = 1 ;
    currentSpacesDims2 [1] = M_maxNumElements ;

    M_HDF5.createTable ("element_ids", H5T_STD_U32BE, currentSpacesDims2) ;
    M_HDF5.createTable ("element_nodes", H5T_STD_U32BE, currentSpacesDims) ;

    M_uintBuffer.resize (currentSpacesDims[0]*currentSpacesDims[1], 0) ;
    std::vector<size_type> idsBuffer ;
    idsBuffer.resize (currentSpacesDims2[1], 0) ;

    M_uintBuffer.resize (currentSpacesDims[0]*currentSpacesDims[1], 0) ;

    size_type k = 0 ; //counter for partitions
    size_type i = 0 ; //counter for elements
    for (p_it = M_meshOut->beginParts (); k < M_numParts ;  p_it++ , k++)
    {
        auto elt_it = M_meshOut->beginElementWithMarker (p_it->first) ;
        auto elt_en = M_meshOut->endElementWithMarker (p_it->first) ;
        for ( ; elt_it != elt_en ; ++elt_it , i ++)
        {
            idsBuffer[i] = elt_it->id () ;
            for ( size_type j = 0 ; j < M_elementNodes ; j ++ )
                M_uintBuffer[j + M_elementNodes*i] = elt_it->point(j).id() -1 ; 
        }
    }

    hsize_t currentOffset[2] = {0, 0} ;
    M_HDF5.write ( "element_ids", H5T_NATIVE_LLONG, currentSpacesDims2, currentOffset, &idsBuffer[0] ) ;
    M_HDF5.write ( "element_nodes", H5T_NATIVE_LLONG, currentSpacesDims, currentOffset, &M_uintBuffer[0] ) ;

    M_HDF5.closeTable ("element_ids") ;
    M_HDF5.closeTable ("element_nodes") ;
}
\end{lstlisting}
Ce code n'est pas encore optimisé, en effet il faudrait utiliser un buffer différent pour chaque partition et pas les regrouper dans un seul buffer.

\section {Exportation des données sur les noeuds}
Pour l'instant je vois à peu près comment on accède aux données des noeuds, mais je rencontre quelques problèmes techniques avec la classe timeset. Tout d'abord pour exporter les données des solutions générés par feel++, il est inévitable de passer par les "timeset". De ce fait je dois faire hérité ma classe de la classe \emph{Exporter}, en redéfinissant toutes les méthodes \emph{virtual} (ces méthodes ont des corps vide pour le moment). Ensuite il faut que je trouve les informations suivantes :
\begin{enumerate} 
\item la durée de la simulation
\item le pas de temps
\item comment stocker les données dans le fichier hdf5 pour chaque pas de temps 
\end{enumerate}


\end{document}
